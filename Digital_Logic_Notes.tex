\documentclass{article}
\usepackage{booktabs}
\newcommand{\itab}[1]{\hspace{0em}\rlap{#1}}
\newcommand{\tab}[1]{\hspace{.2\textwidth}\rlap{#1}}

\begin{document}

\title{Fancy Digital Logic Notes in \LaTeX{}}
\author{Adeeb Kabir}

\maketitle

\begin{abstract}This document will detail notes on digital logic taken at Rutgers New Brunswick CS211 (Computer Architecture) with Brian Russell.

\end{abstract}

\section{Transistors}
Transistors are semiconductive devices that amplify and switch electronic devices.

They are:
\begin{tabbing}
	\tab{1.\ used in electrical switches without moving parts}\\

	\tab{2.\ embedded into integrated circuits.}\\

	\tab{As such, they are the building blocks of computers.}
\end{tabbing}

An aside from Professor Russell: `Superconductors change resistance when cooled, but semiconductors change resistance when heated.'\\


\subsection{MOS (Metal Oxide Semiconductors)}

These generally have four terminals:

\begin{tabbing}
	\tab{Source: from which current flows}\\

	\tab{Drain: towards where the current flows}\\

	\tab{Gate:  has 0 or positive voltage}\\

	\tab{Body: substrate (underlying material)}
\end{tabbing}


While not on the exam, this might be extra credit.  This is more electrical engineering.

\section{Logic Gates}
We use these to facilitate boolean logic in circuits.\
Gates include AND, NAND, OR, NOR, NOT, etc.

\begin{tabbing}
	\tab{ABC = $A\wedge B\wedge C$   (Note: $\wedge$ means `AND')}\\

	\tab{A + B + C = $A\vee B\vee C$   (Note: $\vee$ means `OR')}\\

	\tab{$\overline{A}$ = $\neg A$ = NOT A}
\end{tabbing}

For \textit{n} inputs, build a truth table with $2^{n}$ lines to account for all input possibilities.

AND looks sort of like multiplication.  This makes sense, if you look at the inputs:
\begin{tabbing}
	\tab{1 AND 1 is like 1(1) = 1}\\
	\tab{1 AND 0 is like 0(1) = 0}\\
	\tab{0 AND 0 is like 0(0) = 0}
\end{tabbing}

OR looks sort of like addition.  Sorta.
\begin{tabbing}
	\tab{1 OR 1 is like 1 + 1 = 1 (on OR on must certainly mean on)}\\
	\tab{1 OR 0 is like 1 + 0 = 1 }\\
	\tab{0 OR 0 is like 0 + 0 = 0 }
\end{tabbing}

We combine logic gates into full circuits.  We can test them with truth tables.  Let's look at an example.  

We're given a circuit's outputs given its inputs.  We can model its behavior as such:

\begin{center}
	\begin{tabular}{ccc}
		\toprule
		\multicolumn{2}{c}{Input} & Output \\
		\midrule
		0 & 0 & 1 \\
		0 & 1 & 0 \\
		1 & 0 & 1 \\
		\bottomrule
	\end{tabular}

	\bigskip
	\bigskip
	\bigskip

	\begin{tabular}{ccccc}
		\toprule
		\multicolumn{5}{c}{A Given Truth Table}\\
		\cmidrule{1-5}
		A & B & C & \[ F_{1}\] & \[ F_{2}\] \\
		\midrule
		0 & 0 & 0 & 1 & 1 \\
		0 & 0 & 1 & 1 & 1 \\
		0 & 1 & 0 & 0 & 1 \\
		0 & 1 & 1 & 1 & 0 \\
		1 & 0 & 0 & 0 & 1 \\
		1 & 0 & 1 & 0 & 1 \\
		1 & 1 & 0 & 0 & 0 \\
		1 & 1 & 1 & 1 & 0 \\
		\bottomrule
	\end{tabular}
\end{center}
Due to this table, we can say \[ F_{1} = \bar{A} \bar{B} \bar{C} + \bar{A} \bar{B} C + ABC \]
This is called a \textbf{Sum of Expressions} form.

\subsection{Algebraic rules for logical expressions}
\begin{center}

	\bigskip

	\begin{tabular}{lll}
		\toprule
		Rule & Form in Summation & Form in Multiplication\\
		\midrule
		Commutative & $w + y = y + w$  & $wy = yw$ \\
		Associative & $(w + y) + z = w + (y + z)$ & $w(yz) = (wy)z$ \\
		Distributive & $w + yz = (w + y)(y + z)$ & $w(y + z) = (wy + wz)$ \\
		Indempotent & $w + w = w$ & $ww = w$ \\
		Involution & $\overline{\overline{w}} = w$\\
		Complement & $w + \overline{w} = 1$ & $w \overline{w} = 0$\\
		\bottomrule
	\end{tabular}

	\bigskip
	\bigskip
	\bigskip

	\begin{tabular}{ll}
		\toprule
		Identities\\
		\midrule
		$1 + w = 1$ & $0w = 0$\\
		$0 + w = w$ & $1w = w$\\
		$w + \overline{w} y = w + y$\\
		\bottomrule
	\end{tabular}

	\bigskip

\end{center}

\section{Conclusion}
Using this sort of understanding of circuits as logic gates, and thus logical expressions, we can simplify complex circuits into equivalent ones using logical rules.\\

Thaaaaat being said, this is still more an electrical engineering endeavor.

\end{document}

